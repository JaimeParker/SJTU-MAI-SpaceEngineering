\documentclass[conference]{IEEEtran}
\IEEEoverridecommandlockouts
% The preceding line is only needed to identify funding in the first footnote. If that is unneeded, please comment it out.
\usepackage{cite}
\usepackage{amsmath,amssymb,amsfonts}
\usepackage{algorithmic}
\usepackage{graphicx}
\usepackage{textcomp}
\usepackage{xcolor}
\def\BibTeX{{\rm B\kern-.05em{\sc i\kern-.025em b}\kern-.08em
    T\kern-.1667em\lower.7ex\hbox{E}\kern-.125emX}}


\begin{document}

\title{Satellite Position Calculation\\
{\footnotesize \textsuperscript{*}Note:
homework 3 of Navigation Principle of SJTU}
}

\author{\IEEEauthorblockN{Liu Zhaohong}
\IEEEauthorblockA{Student ID: 122431910061}
}

\maketitle

\begin{abstract}
Based on the Yuma and Rinex files, calculating satellite PRN-01’s position in both orbital and ECEF coordinates. Rinex file is version 2.
\end{abstract}

\begin{IEEEkeywords}
orbit, satellite, position, SV, ECEF
\end{IEEEkeywords}

\section{Introduction}
There are several key points. How to get useful information from `Yuma` file and `Rinex` data file? How to get satellite's SV position(or coordinate parameters) from these useful data? How to convert SV format position to ECEF format position(there should be a transformation matrix with 3 variables, 3 angles)

\section{Read Yuma and Rinex Data}

\subsection{Yuma Information}

According to Yuma definition,
\begin{itemize}
	\item ID, PRN of the SVN, namely satellite ID of GPS satellites
	\item Health
	\item Eccentricity, which is $e$
	\item Time of Applicability, the number of seconds in the orbit when the almanac was generated. Kind of a time tag
	\item Orbital Inclination, The angle to which the SV orbit meets the equator (GPS is at approximately 55 degrees)
	\item Argument of Perigee
	\item Mean Anomaly
	\item Week
\end{itemize}

\subsection{Rinex Information}

too complex


\section{Position Acquiring and Coordinate Converting}

Once we know what we can acquire from the two data file, we shall consider how to use these data to realize our function.

There are six parameters about the orbit, 3 about the satellite orbit and 3 about the coordinate conversion.

\subsection{Position in Satellite Orbit}

To describe a orbit, we need know the eccentricity, and a; also a Perigee degree to gain satellite's angle relative to the focal point.

Based on Yuma data, we can get these,
\begin{itemize}
	\item $\sqrt{a}=5153.616211$
	\item $e=0.01200914383$
	\item $v=0.944318348 \quad \text{rad}$
	\item $\Omega = 0.2426539643 \quad \text{rad}$
	\item $i = 0.9889106755 \quad \text{rad}$
	\item $\omega =  -0.7640576294 \quad \text{rad}$
\end{itemize}

Position in orbit coordinate has no $z$ axis values, because satellite move in a plane \textbf{XOY}. so we can get it's $x$ and $y$ by the $v$:

\begin{equation}
	\begin{cases}
	x = r \cos v\\
	y = r \sin v
	\end{cases}
\end{equation}

and

\begin{equation}
r(v)=\frac{a(1-e^2)}{1+e\cos v}
\end{equation}

Then the $x$ and $y$ coordinate in satellite axis is acquired.

\subsection{Coordinate Transformation}

It is easy to get the position in satellite coordinate, so we will compose a way to convert position in satellite axis to ECEF.

from SV ECEF to SV, it a process of 3 times rotation:
\begin{itemize}
	\item $\Omega$ around axis z
	\item $i$ around axis x
	\item $\omega$ around axis z
\end{itemize}

so it's not complex to get a rotation matrix form SV to ECEF, so as ECEF to SV.

According to known formula on class,

\begin{equation}
X_T=
	\begin{bmatrix}
	x_p\cos\Omega_p - y_p\sin \Omega_p \cos i\\
	x_p\sin\Omega_p + y_p\cos \Omega_p \sin i\\
	y_p \sin i\\
	\end{bmatrix}
\end{equation}

\subsection{Take Time into Consideration}

setting time of Yuma as origin, time ascending will cause change in $v, \Omega, \omega$


\begin{thebibliography}{00}
\bibitem{b1}https://www.mathworks.com/matlabcentral/fileexchange/73873-satellite-orbit-coordinate-transformations
\bibitem{b2}https://juliaspace.github.io/SatelliteToolbox.jl/stable/man/orbit/tle/
\bibitem{b3}https://indico.ictp.it/event/a12180/session/14/contribution/8/material/0/0.pdf
\bibitem{b4}https://celestrak.org/GPS/almanac/Yuma/definition.php
\bibitem{b5}https://en.wikipedia.org/wiki/Ellipse
\bibitem{b6}https://oc.sjtu.edu.cn/courses/47270/files/folder

\end{thebibliography}

\end{document}
